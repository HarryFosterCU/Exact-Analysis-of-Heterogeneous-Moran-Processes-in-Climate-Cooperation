\documentclass{article}

\usepackage[margin=1.5cm, includefoot, footskip=30pt]{geometry}
\usepackage{amsmath}
\usepackage{amsfonts}
\usepackage{amssymb}

\title{General Heterogeneous Moran Processes}
\author{Vince Knight and Harry Foster}
\date{\today}

\begin{document}
\maketitle
\section{Introduction}

\section{Literature Review}

\section{Model}



In this section, we will look at the underlying model of the heterogeneous Moran process. We define the following:

\begin{itemize}
    \item \(N\) \textbf{ordered} individuals
    \item \(k\) types \(A_1, A_2, \dots A_k\)
    \item A state space S given by the set of ordered N-tuples with entries of type \(A_1, A_2, \dots A_k\). 
          Note that \(|S|=k^N\)
    \item A fitness function \(f:S \to \mathbb{R}^N\)
\end{itemize}

Let $h(V,U)$ denote the Hamming distance between two states V and U. We consider a Markov chain, where for the transition $V = (v_1, v_2, \dots, v_N) \to U = (u_1, u_2, \dots, u_N)$, the 
transition probability is defined as follows:\\
\begin{large}
    \begin{equation}
        P(V,U) = 
        \begin{cases}
            \frac{\sum_{v_i = u_{i*}}{f(v_i)}}{\sum_{v_i}f(v_i)} & \text{if h(V,U) = 1, differing at position i*}\\
            0 & \text{if h(V,U) \textgreater 1}\\
            1 - \sum_{U \in S \setminus \text{\{V\}}}{P(V,U)} & \text{if h(V,U) = 0}
        \end{cases}
    \end{equation}
\end{large}

The final case is given by the following notion. V can transition to itself through the removal of any constituting individual, and the duplication of any individual which shared a type with the removed one. Thus, directly calculating P(V,V) would require a large amount of computational time and effort. However, we can observe that a transition of some sort must occur, and so we simply take the probability of not transitioning to any \textbf{different} state as P(V,V).\\

An important case which proceeds from (1) is that of a transition V $\to U$ where U contains an individual of a \textit{type} not found in V. For example, the transition (0,1) $\to$ (0,2). This would be forbidden by the intuition of a Moran process - i.e the new individual being the duplication of another individual in V. However, we can see that the standard formula yields P(V,U) = 0 because $\nexists v_i \in V: v_i = u_{i*}$. Following this, we see that (1) also correctly gives us P(V,U) = 0 for any \textbf{absorbing} V $\neq$ U.




\end{document}