\documentclass{article}

\usepackage[margin=1.5cm, includefoot, footskip=30pt]{geometry}
\usepackage{amsmath}
\usepackage{amsfonts}
\usepackage{amssymb}

\title{General Heterogeneous Moran Processes}
\author{Vince Knight and Harry Foster}
\date{\today}

\begin{document}
\maketitle
\section{Introduction}

\section{Literature Review}

\section{Model}



In this section, we will look at the underlying model of the heterogeneous Moran process. We define the following:

\begin{itemize}
    \item \(N\) \textbf{ordered} individuals
    \item \(k\) types \(A_1, A_2, \dots A_k\)
    \item A state space S given by the set of ordered \(N\)-tuples with entries of type \(A_1, A_2, \dots A_k\). 
          Note that \(|S|=k^N\).\\
    A state \(v = (v_1, v_2, \dots) \in S\) is called \textbf{absorbing} if \(v_i = v_j\) \(\forall\) \(i,j \in \mathbb{N^+}\) : \(i,j \le N\)
    \item A fitness function \(f:S \to \mathbb{R}^N\)
\end{itemize}

Let $h(v,u)$ denote the Hamming distance~\cite{Hamming_Distance} between two states \(v, u \in S\). We consider a Markov chain~\cite{Markov_Chain}, where for the transition $V = (v_1, v_2, \dots, v_N) \to U = (u_1, u_2, \dots, u_N)$, the 
transition probability is defined as follows:\\
\begin{large}
    \begin{equation}
        p_{v, u} = 
        \begin{cases}
            \frac{\sum_{v_i = u_{i^*}}{f(v_i)}}{\sum_{v_i}f(v_i)} & \text{if }h(v,u) = 1 \text{, differing at position }i^*\\
            0 & \text{if }h(v,u) > 1\\
            1 - \sum_{u \in S \setminus \text{\{v\}}}{P_{v,u}} & \text{if }h(v,u) = 0
        \end{cases}
    \label{eqn:(1)}
    \end{equation}
\end{large}

The final case corresponds to a transition from \(v\) to itself. 
This is possible when a given individual has their action type removed
and any other individual with the same type is chosen for duplication. A direct computation of $P_{v,v}$ is given by:
\begin{large}
\begin{equation}
    \frac{1}{N}\sum_{i=1}^{N} \frac{\sum_{v_j = v_{i}}{f(v_i)}}{\sum_{v_j}f(v_i)}
    \label{eqn:(2)}
\end{equation}
\end{large}
This would require at least \(N^2 + N\) calculations. 
However, we can observe that a transition of some sort must occur, and so we simply take the probability of not transitioning to any \textbf{different} state as \(P_{v,v}\).\\

An important case which proceeds from (\ref{eqn:(1)}) is that of a transition \(v \to u\) where \(u\) contains an individual of a \textit{type} not found in \(v\). For example, the transition (0,1) $\to$ (0,2). This would be forbidden by the intuition of a standard Moran process - i.e the new individual being the duplication of another individual in \(v\). However, we can see that the standard formula yields \(p_{v,u}\) = 0 because \(\nexists\) \(v_i \in V: v_i = u_{i^*}\). Following this, we see that (\ref{eqn:(1)}) also correctly gives us \(p_{v,u}\) = 0 for any \textbf{absorbing} \(v \neq u\).

\begin{thebibliography}{9}
\bibitem{Hamming_Distance}
Dave K. Kythe, Prem K. Kythe (2012) - Algebraic and Stochastic Coding Theory
\bibitem{Markov_Chain}
William J. Stewart (2009) - Probability, Markov Chains, Queues, and Simulation: The Mathematical Basis of Performance Modeling


\end{thebibliography}


\end{document}