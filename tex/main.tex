\documentclass{article}

\usepackage[margin=1.5cm, includefoot, footskip=30pt]{geometry}
\usepackage{amsmath}
\usepackage{amsfonts}
\usepackage{amssymb}
\usepackage{graphicx}

\title{A Heterogeneous Moran Process for the Analysis of Public Goods Games}
\author{Vince Knight and Harry Foster}
\date{\today}

\begin{document}
\maketitle
\section{Introduction}

\section{Literature Review}
Evolutionary game theory has long been at the forefront of the study of
emergent cooperation, showing how a strategy that is seemingly unfavourable for
the individual, but beneficial to the collective, can become dominant in
systems of rational actors. This is often shown through the study of games such
as the iterated prisoner's dilemma\cite{IPD1,IPD2} and the snowdrift
game\cite{Snowdrift1}. These examples, however, only allow for pairwise
interactions, which does not accurately represent many real-world scenarios.
Therefore, we look at a common game used to model the sharing of resources and
interactions between multiple people - the public goods game.\\

In order to further tailor the idea of a public goods game to real world
scenarios, we introduce the idea of heterogeneity to the system. Often in the
study of games we consider each individual to exist in the same conditions -
having the same utility and probability of transitioning to a different action
type. However, this does not accurately reflect many of the interactions that
occur in the real world, where individuals may have different factors affecting
which strategies they are able to take. For example, in a public goods game,
one player may wish to contribute \(x\) units to the public good, however they
do not have enough wealth available to spare such a contribution. Heterogeneity
allows us to give varying attributes to the different actors in order to
simulate such situations. These attributes can be encoded into the players
themselves as separate to the game, for example the reputation in \cite{Maa},
exist within the strategies of the players \cite{Lei, Flores}, or they may
exist within the actual payoffs which the players receive. We will look at
multiple different ways in which heterogeneity can be implemented in order to
model different scenarios.\\

In much of the literature, we see heterogeneous public goods game being played
graphically. By this I mean that in much research, players are placed on graphs
(often regular square lattices, as in \cite{Maa, Flores}) and participate in
multiple public goods games at the same time, with their payoff being the
combined payoff from all their games. This gives even more weight to the
decision to cooperate or defect, and allows players to interact in scenarios
with many different levels of these heterogeneous attributes. We also see many
ideas for what these attributes could be based upon. \cite{Maa} takes the idea
of that players could build a reputation through their decisions across
multiple generations, with players refusing to contribute higher amounts to
groups made up of those who have a reputation for defection. While this is a
dynamically updating heterogeneous attribute, some have proposed static
attributes based on inherent properties of the individuals, such as in
\cite{Lei} where the players participate in different amounts of games, and
contribute different amounts based on this factor.\\

In both of the above cases, it is shown that an increased scrutiny of heterogeneous attributes in a spatial public goods game is beneficial to the emergence of cooperation, though for different reasons. In \cite{Maa}, we see that increasing such scrutiny obviously harms defectors, as they receive reduced payoffs based on their low reputation due to being unable to coerce high contributions from other players. \cite{Lei}, however, finds that if we encourage players to cooperate with those in a similar number of games to them, the high-degree defector nodes are unable to gather a large payoff, and so will be unable to spread their defection to their neighbours. If you encourage players to interact with those with different degrees, however, then middle-degree defectors will make heavy losses and be unable to survive.\\

\cite{Flores} shows how the benefits of this heterogeneity are not always as pronounced as it may seem. When the contribution is not calculated by some attribute, but rather is an attribute in and of itself (essentially becoming a new strategy) then even in systems where contribution dominates, we will see low-value contributions act as a sort of defection against the most globally beneficial strategies. Therefore, we can see that the emergence of cooperation is not the end of the story in some cases, and we must look at what sort of cooperation it is that we have fostered the emergence of.\\

One common theme in the spatial public goods game is the manner in which cooperation spreads. An initial ``invasion" of defectors leads to cooperation mainly existing within small clusters. These clusters, however, are much more profitable than their defecting neighbours. Therefore, the cooperating clusters will influence the nearby defectors to join in, which eventually spreads throughout the system. This relies on the ability of the clusters to resist the initial invasion of defectors - an occurrence which relies on a high enough \(r\) value (the positive multiplier of the contribution to the public good).\\

We also see that a potential climate club may soon be forming within the European Union. While it functions slightly differently to the classic example in \cite{Climate Clubs}, it follows the same principles of encouraging countries to join the group by punishing those outside of the club, in order to attempt to reduce overall carbon emissions. This is the ``Carbon Border Adjustment Mechanism" \cite{CBAM}. Currently, all goods within the EU require taxation based on their carbon footprint, however imported goods from outside cannot be subject to such taxes. CBAM aims to prevent foreign goods from gaining benefits due to environmentally harmful practices by enforcing that EU companies report the carbon content of imported goods, and purchase certificates to cover the environmental cost of said goods. However, if a country already levies a tax on their own goods (similar to the EU's carbon tax), then this cost will be taken into consideration, and fewer certificates must be purchased. This discourages the importation of goods from countries which do not levy a similar tax on their own items, forming a structure which behaves very similarly to the ``climate clubs" in \cite{Climate Clubs}.




\section{The Model}

\subsection{An Initial System}

In this section, we will look at the underlying model of the heterogeneous Moran process. We define the following:

\begin{itemize}
    \item \(N\) \textbf{ordered} individuals
    \item \(k\) types \(A_1, A_2, \dots A_k\)
    \item A state space S given by the set of ordered \(N\)-tuples with entries of type \(A_1, A_2, \dots A_k\). 
          Note that \(|S|=k^N\).\\
    A state \(v = (v_1, v_2, \dots) \in S\) is called \textbf{absorbing} if \(v_i = v_j\) \(\forall\) \(i,j \in \mathbb{N^+}\) : \(i,j \le N\), and the set of absorbing states is called \(S^\Gamma\)
    \item A strictly positive fitness function \(f:S \to \mathbb{R}^N\)
\end{itemize}

Let $h(v,u)$ denote the Hamming distance~\cite{Hamming_Distance} between two states \(v, u \in S\). We consider a Markov chain~\cite{Markov_Chain}, where for the transition $v = (v_1, v_2, \dots, v_N) \to u = (u_1, u_2, \dots, u_N)$, the 
transition probability is defined as follows:\\
\begin{large}
    \begin{equation}
        p_{v, u} = 
        \begin{cases}
            \frac{\sum_{v_i = u_{i^*}}{f(v_i)}}{\sum_{v_i}f(v_i)} & \text{if }h(v,u) = 1 \text{, differing at position }i^*\\
            0 & \text{if }h(v,u) > 1\\
            1 - \sum_{u \in S \setminus \text{\{v\}}}{P_{v,u}} & \text{if }h(v,u) = 0
        \end{cases}
    \label{eqn:Probability_Function}
    \end{equation}
\end{large}

Transitioning denotes something slightly different in the heterogeneous case of the Moran process as compared with the homogeneous. While we would traditionally say an \textit{individual} was removed and a new individual was generated to replace them, we must now say an individual's \textit{action type} was removed, and replaced with a new action type. This is because the individuals themselves are ordered in the heterogeneous Moran process, and possess certain attributes specific to themselves. Therefore, we cannot say that the individual themself is replaced, but rather that their action type is replaced.\\

The final case in (\ref{eqn:Probability Function}) corresponds to a transition from \(v\) to itself. 
This is possible when a given individual has their action type removed
and any other individual with the same type is chosen for duplication. A direct computation of $P_{v,v}$ is given by:
\begin{large}
\begin{equation}
    \frac{1}{N}\sum_{i=1}^{N} \frac{\sum_{v_j = v_{i}}{f(v_i)}}{\sum_{v_j}f(v_i)}
    \label{eqn:(2)}
\end{equation}
\end{large}
This would require at least \(N^2 + N\) calculations. 
However, we can observe that a transition of some sort must occur, and so we simply take the probability of not transitioning to any \textbf{different} state as \(P_{v,v}\).\\

An important case which proceeds from (\ref{eqn:Probability Function}) is that of a transition \(v \to u\) where \(u\) contains an individual of a \textit{type} not found in \(v\). For example, the transition (0,1) $\to$ (0,2). This would be forbidden by the intuition of a standard Moran process - i.e the new individual being the duplication of another individual in \(v\). However, we can see that the standard formula yields \(p_{v,u}\) = 0 because \(\nexists\) \(v_i \in V: v_i = u_{i^*}\). Following this, we see that (\ref{eqn:Probability}) also correctly gives us \(p_{v,u}\) = 0 for any \textbf{absorbing} \(v \neq u\).\\

This model allows for heterogeneity as we have a fitness function dependent on the individual, not dependent on the action taken. Therefore, we can give different individuals of the same action type a different fitness by passing different attributes to the fitness function itself. We will see examples of this in future sections.\\

As this forms a Markov chain, we can therefore define a transition matrix \(T\) for any given state space. This can be used to show the transition probabilities between any two given states, and therefore can be used to find the fixation probabilities for any given starting state by the following method:\\

Given a transition matrix \(T\), we can calculate the state distribution after \(k\) iterations of the Moran process using the formula \(T^k\)  \cite{Markov_Chain}. Therefore, as \(k \to \infty\), we will clearly acquire a distribution that tends towards the absorption probabilities for a given starting state, as the process will stabilise only if we enter an absorbing state (while subsets of \(S\) of states may have extremely high probabilities of transitioning to each other, the only way to have certainty of not transitioning away from a state or pair of states is to be in one of the previously defined absorbing states.)\\
Thus, our absorption probabilities starting at are given by the non-zero entries in the matrix \(\lim_{k \to \infty} T^k\)

\section{The General Heterogenous Public Goods Game}
\subsection{The Standard Model and Transformations}

In this section, we shall see how this model can be applied to the classical public goods game. In this game, individuals choose whether or not to contribute to a public resource pool. In the end, the total resource is multiplied by some factor \(r > 1\) and distributed equally between each player. This model often encourages a behaviour known as ``free-riding" \cite{Climate Clubs}, where players refuse to contribute to the pool and simply take the benefit provided by other individuals' contribution. The classic problem of a public goods game is to provide an environment where such free-riding is not a profitable strategy - or rather, is less profitable than contributing to the public good.\\

In our model, we can simulate a public goods game by providing the following payoff function:\\
\begin{large}
\begin{equation}
    \sigma(v_i) = \frac{r\sum_{j=1}^N{C_{v_j}}}{N} - C_{v_i}
    \label{eqn:General Model}
\end{equation}
\end{large}\\
(where \(C_{v_j}\) is the contribution by individual j.) 

In the homogeneous public goods game, we can see that \(C_{v_j}\) can only take one of two values: some constant \(\alpha \) if individual j cooperates, and 0 if not. However, the heterogeneous game will require a more tailored \(C_{v_j}\).\\

Note that \(\sigma(v_j)\) can be negative which would then not lead to sensible probability values as needed in~\ref{eqn:Probability Function}. Therefore, we must look at some transformation of \(\sigma\) to use as our fitness function.\\

A common method \cite{Nowak 2006} for this is to apply the exponential function to our payoff function. By using \(e^{\kappa \sigma}\) as our fitness function we will have positive values, however this particular approach  gives:

\begin{large}
    \begin{equation}
        \frac{\sum_{v_i = u_{i^*}}{e^{\frac{\kappa r\sum_{j=1}^N{C_{v_j}}}{N} - C_{v_i}}}}{\sum_{v_i}e^{\frac{\kappa r\sum_{j=1}^N{C_{v_j}}}{N} - C_{v_i}}}
    \end{equation}
\end{large}\\

Now, as \(e^r\) does not rely on \(v\), we can take this out of the sum and acquire the following:

\begin{large}
    \begin{equation}
        \frac{e^{\frac{\kappa r\sum_{j=1}^N{C_{v_j}}}{N}}\sum_{v_i = u_{i^*}}{e^{ - C_{v_i}}}}{e^{\frac{\kappa r\sum_{j=1}^N{C_{v_j}}}{N}}\sum_{v_i}e^{-C_{v_i}}} = \frac{\sum_{v_i = u_{i^*}}{e^{ - C_{v_i}}}}{\sum_{v_i}e^{-C_{v_i}}}
    \end{equation}
\end{large}\\

and we see that in this case, \(p_{v,u}\) would not rely on \(r\). 
Therefore, let us consider some other methods of guaranteeing a positive fitness function.\\

The first of these is known as the "shifted linear" transformation. We take, for some small tunable \(\epsilon\):
\begin{large}
    \begin{equation}
        f(v_i) = \sigma(v_i) - \min_{v_j \in v} \sigma(v_j) + \epsilon
    \end{equation}
    \label{eqn:Shifted_linear}
\end{large}
This guarantees a positive value, with the lowest fitness taking the value \(\epsilon\), a parameter that can be chosen based on the system in question.\\

Another type of mapping that we can use is the ``Affine-linear mapping". In
this, for some tunable \(\epsilon\), we take the following:

\begin{large}
    \begin{equation}
        f(v_i) = 1 + \epsilon\sigma(v_i)
    \end{equation}
    \label{eqn:Affine-linear}
\end{large}\\




\subsection{The Homogeneous Case}
The most common type of public goods game is the homogeneous public goods game.
In this game, we consider that we have a constant \(C_{v_i} = \alpha\) for all players \(v_i\).
In this case, we take the payoff function:
\begin{large}
    \begin{equation}
        \sigma(v_i) = \frac{r\sum_{j=1}^N{k_j\alpha}}{N} - k_i\alpha = (Kr-k_i)\alpha
    \end{equation}
    \label{eqn:Homogeneous_fitness}
\end{large}\\
Where \(k_j = 1\) if j contributes and 0 if not, and \(K\) is the fraction of the population who contribute.\\

However, we cannot use \ref{eqn:Shifted_linear} transformation or else we once again run into a problem with too many factors cancelling down and becoming unreliant on \(r\):

\begin{large}
    \begin{equation}
        \begin{split}
        (K * r - k_i)\alpha - \min_j(K * r  - k_j)\alpha + \epsilon 
        = (1 - k_i)\alpha + \epsilon = \begin{cases}
            \epsilon & \text{if \(v_i\) contributes}\\
            \alpha + \epsilon & \text{if not}
        \end{cases}
        \end{split}
    \end{equation}
    \label{eqn:Homogeneous_fitness}
\end{large}\\

Not only does our choice of \(\epsilon\) become too powerful, but also we cancel out all reliance on \(r\)\\

Therefore, we take the affine-linear mapping shown in \ref{eqn:Affine-linear} in order to see our fixation probabilities.

We will begin by looking at the state space with \(N=4\) and see how cooperation emerges based on different values of \(r \text{ and } \alpha\).

\begin{figure}[!h]
    \centering
\includegraphics[width=1\linewidth]{./Figures/Homogeneous_Fixation_Probabilities/Main.png}
    \caption{Fixation probabilities in the homogeneous case. Here we take }
    \label{fig:Homogeneous_Fixation_Probability_Examples}
\end{figure}

\newpage
\section{Public Goods Games with Heterogeneous Returns}

When we look at a public goods game, we often generalise the players to all receive the same return. However, in the real world this is often not the case. 


\newpage
\begin{thebibliography}{9}
\bibitem{IPD1}
Vince Knight, Marc Harper, Nikoleta Glynatsi et al (2024) - Recognising and evaluating the effectiveness of extortion in the Iterated Prisoner’s Dilemma
\bibitem{IPD2}
Robert Axelrod (1980) - Effective Choice in the Prisoner’s Dilemma
\bibitem{Snowdrift1}
%Zhenyu Wang & Nigel Goldenfeld (2011) - Theory of cooperation in a micro‑organismal snowdrift game
\bibitem{Maa}
Xiaoiian Maa et al (2021) - Effect of reputation-based heterogeneous investment on cooperation in spatial public goods game
\bibitem{Lei}
Chuang Lei et al (2010) - Heterogeneity of allocation promotes cooperation in public goods games
\bibitem{Flores}
Lucas S. Flores et al (2023) - Heterogeneous contributions can jeopardize cooperation in the public goods game
\bibitem{Hamming_Distance}
Dave K. Kythe, Prem K. Kythe (2012) - Algebraic and Stochastic Coding Theory
\bibitem{Markov_Chain}
William J. Stewart (2009) - Probability, Markov Chains, Queues, and Simulation: The Mathematical Basis of Performance Modeling
\bibitem{Climate Clubs}
William Nordhaus (2015) - Climate Clubs: Overcoming Free-riding in International Climate Policy
\bibitem{Nowak 2006}
Martin A. Nowak (2006) - Evolutionary Dynamics: Exploring the Equations of Life
\bibitem{CBAM}
Regulation (EU) 2023/956 of the European Parliament and of the Council of 10 May 2023 establishing a carbon border adjustment mechanism (Text with EEA relevance)
\end{thebibliography}


\end{document}